
\documentclass{article}
\usepackage[a4paper, top=1cm, right=2cm, bottom=1cm, left=2cm]{geometry}

\usepackage[utf8]{inputenc}
\usepackage[T1]{fontenc}
\usepackage{amsmath}
\usepackage{tikz}

\title{\vspace{-2em}\Huge{\textbf{Chap 6 : Graphes}}\vspace{-1.5em}}
\author{}
\date{}

\begin{document}
    \maketitle

    \section{Introduction aux graphes}
        \subsection{Vocabulaire}
            Définition:\\
            \framebox[\textwidth][l]{Un graphe est un ensemble de sommets}
        
        \subsection{Graphes complets}
        \subsection{Graphes non-orientés et chaînes}
            \subsubsection{Chaînes}
            \subsubsection{Chaîne d'Euler}


    \section{Graphes orientés et lien avec les matrices}
        \subsection{Graphes orientés}
        \subsection{Matrices}
            \subsubsection{Matrice d'adjacence}
            \subsubsection{Puissances de matrices}


    \section{Pour aller plus loin}
        \subsection{Chaîne de Markov}

        \begin{tikzpicture}[node distance={15mm}, thick, main/.style = {draw, circle}]
            
        \end{tikzpicture}




\end{document}
